%%%%%%%%%%%%%%%%%%%%%%%%%%%%%%%%%%%%%%%%%%%%%%%%%%%%%%%%%%%%%%%%%%%%%%%%
%%%%%%%%%%%%%%%%%%%%%%%%%%%%%%%%%%%%%%%%%%%%%%%%%%%%%%%%%%%%%%%%%%%%%%%%
\section{ZKPs of knowledge vs.\ ZKPs of membership}
\label{security:zkp-knowledge-vs-membership}


	The theory of ZKPs distinguishes between two types of proofs, based on the type of statement (and also on the type of security properties --- see Sections \ref{sec:security:defs-props:soundness} and \ref{sec:security:defs-props:proof-of-knowledge}):
	
\begin{itemize}
\item A ZKP of knowledge (ZKPoK) proves the veracity of a \emph{statement of knowledge}, i.e., it proves knowledge of private data that supports the statement, without revealing the former.
\loosen

\item A ZKP of membership proves the veracity of a \emph{statement of membership}, i.e., that the \emph{instance} belongs to the \emph{language}, as related to the \emph{statement}, but without revealing information that could not have been produced by a computationally bounded verifier.
\end{itemize}


	The \emph{statements} exemplified in \reftab{tab:example-scenarios-zkps} were expressed as facts, but each of them corresponds to a knowledge of a secret witness that supports the statement in the context of the instance.
	For example, the statement ``I am an adult'' in scenario 1 can be interpreted as an abbreviation of ``I know a birthdate that is consistent with adulthood today, and I also know a certificate (signed by some trusted certification authority) associating the birthdate with my identity.''


    The first three use-cases (adulthood, solvency and asset ownership) in \reftab{tab:example-scenarios-zkps} have instances with some kind of protection, such as physical access control,  encryption, signature and/or commitments.
	The ``chessboard configuration'' and the ``theorem validity'' use-cases are different in that their instances do not contain any cryptographic support or physical protection. 
	Each of those two statements can be seen as a claim of membership, in the sense of claiming 
that the expression/configuration belongs respectively to the language of valid chessboard configurations (i.e., reachable by a sequence of moves), or the language of theorems (i.e., of provable expressions).
	At the same time, a further specification of the statement can be expressed as a claim of knowledge of a sequence of legal moves or a sequence of logical implications. 



%%%%%%%%%%%%%%%%%%%%%%%%%%%%%%%%%%%%%%
\subsection{Example: ZKP of knowledge of a discrete logarithm (discrete-log)}
\label{security:zkp-knowledge-vs-membership:ZKPoK-DL}

	
	Consider the classical example of proving knowledge of a discrete-log \cite{1990:eurocrypt-89:schnorr}.
	Let $p$ be a large prime (e.g., with 4096 bits) of the form $p=2q+1$, where $q$ is also a prime.
	Let $g$ be a generator of the group $\mathbb{Z}^{\ast}_p = \{1,...,p-1\} = \left\{g^i:i=1,...,p-1\right\}$ under multiplication modulo $p$.
	Assume that it is computationally infeasible to compute discrete-logs in this group, and that the primality of $p$ and $q$ has been verified by both prover and verifier.
	Let $w$ be a secret element (the witness) known by the prover, and let $x = g^w (\textmd{mod }p)$ be the instance known by both the prover and verifier, corresponding to the following statement by the prover: ``I know the discrete-log (base $g$) of the instance ($x$), modulo $p$'' (in other words: ``I know a secret exponent that raises the generator ($g$) into the instance ($x$), modulo $p$'').
	Consider now the relation $R = \left\{(x,w): g^w = x~(\textmd{mod }p)\right\}$.
	In this case, the corresponding language $L = \left\{x: \exists w: (x,w) \in R\right\}$ is simply the set  $\mathbb{Z}^{\ast}_p = \{1, 2, ..., p-1\}$, for which membership is self-evident (without any knowledge of $w$).
    In that sense, a proof of membership does not make sense (or can be trivially considered accomplished with even an empty bit string).
	Conversely, whether or not the prover knows a witness is a non-trivial matter, since the current publicly-known state of the art does not provide a way to compute discrete-logs in time polynomial in the size of the prime modulus (except if with a quantum computer).
	In summary, this is a case where a ZKPoK makes sense but a ZKP of membership does not.
	


%%%%%%%%%%%%%%%%%%%%%%%%%%%%%%%%%%%%%%
\subsection{Example: ZKP of knowledge of a hash pre-image}
\label{security:zkp-knowledge-vs-membership:ZKPoK-hash-pre-image}

	
	Consider a cryptographic hash function $H: \left\{0,1\right\}^{512} \rightarrow \left\{0,1\right\}^{256}$, restricted to binary inputs of length 512. 
	In this definition of $H$, the set of all 256-bit strings is the \emph{co-domain}, which might be a 
super-set of the \emph{image} $L = \left\{H(x):x\in\left\{0,1\right\}^{512}\right\}$ (a.k.a.\ \emph{range}) of $H$.
	Let $w$ be a witness (hash pre-image), known by the prover and unpredictable to the verifier, for some instance $x=H(w)$ that the prover presents to the verifier.
	Since a cryptographic hash function is one-way, there is significance in providing a ZKPoK of a pre-image, which proves knowledge of a witness in the relation $R = \left\{(x,w): H(w)=x\right\}$.
	Such proof also constitutes directly a proof of membership in the language $L$, i.e., that the instance $x$ is a member of the image of $H$.
	However, interestingly depending on the known properties of $H$, this membership predicate might or might not be self-evident from the instance $x$.
\loosen
	
	\begin{itemize}
	\item If $H$ is known to have as image the set of all bit-strings of length 256 (i.e., if $L = \left\{0,1\right\}^{256}$), then membership is self-evident.
	In this case a ZKP of membership is superfluous, since it is trivial to verify the property of a bit-string having 256 bits.
	
	\item $H$ may instead have the property that an element $x$ uniformly selected from the co-domain $\left\{0,1\right\}^{256}$ is not 
in the image of $H$, with some noticeable probability (e.g., $\approx$0.368, if $H$ is modeled as a random function), and with the membership predicate being difficult to determine.
	In this setting it can be useful to have the ability to perform a ZKP of membership.
	\end{itemize}




%%%%%%%%%%%%%%%%%%%%%%%%%%%%%%%%%%%%%%
\subsection{Example: ZKP of membership for graph non-isomorphism}
\label{security:zkp-knowledge-vs-membership:ZKPoM-GNI}

	
	In the theoretical context of provers with super-polynomial computation ability (e.g., unbounded), one can conceive a proof of membership without the notion of witness.
Therefore, in this case the dual notion of a ZKP of knowledge does not apply.
	A classical example uses the language of pairs of non-isomorphic graphs \cite{1991:JACM:proofs-that-yield-nothing-but-their-validity},
for which the proof is about convincing a verifier that two graphs are not isomorphic.
	The classical example uses an interactive proof that does not follow from a witness, but rather from a super-ability, by the prover, in deciding isomorphism between graphs.
	The verifier challenges the prover to detect which of the two graphs is isomorphic to a random permutation of one of the two original graphs.
    If the prover decides correctly enough times, without ever failing, then the verifier becomes convinced of the non-isomorphism.

	This document is not focused on settings that require provers with super-polynomial ability (in an asymptotic setting).
	However, this notion of ZKP of membership without witness still makes sense in other conceivable applications, namely within a concrete setting (as opposed to asymptotic).
	This may apply in contexts of proofs of work, or when provers are ``supercomputers''  or quantum computers, possibly interacting with verifiers with significantly less computational resources.
    Another conceivable setting is when a verifier wants to confirm whether the prover is able to solve a mathematical problem, for which the prover claims to have found a first efficient technique, e.g., the ability to decide fast about graph isomorphism.
 \loosen




