\presection{Executive summary}
\label{sec:prelim:executive-summary}


	

	Zero-knowledge proofs (ZKPs) are an important privacy-enhancing tool from cryptography. 
	They allow proving the veracity of a statement, related to confidential data, without revealing any information beyond the validity of the statement.
	%%% They preserve confidentiality of data, while proving the veracity of statements about the data. %%% statements of membership
	%%% They can also be used to prove knowledge of such data without having to disclose it. %%% Statements of knowledge
  ZKPs were initially developed by the academic community in the 1980s, and have seen tremendous improvements since then.
	They are now of practical feasibility in multiple domains of interest to the industry, and to a large community of developers and researchers.  
  ZKPs can have a positive impact in industries, agencies, and for personal use, by allowing privacy-preserving applications where designated private data can be made useful to third parties, despite not being disclosed to them.


	The development of this reference document aims to serve the broader community, particularly those interested in understanding ZKP systems, making an impact in their advancement, and using related products.
	This is a step towards enabling wider adoption of ZKP technology, which may precede the establishment of future standards.
	However, this document is not a substitution for research papers, technical books, or standards. 
	It is intended to serve as a reference handbook of introductory concepts, basic techniques, implementation suggestions and application use-cases. 


	ZKP systems involve at least two parties: a prover and a verifier.  
	The goal of the prover is to convince the verifier that a statement is true, without revealing any additional information.  
	For example, suppose the prover holds a birth certificate digitally signed by an authority.
	In order to access some service, the prover may have to prove being at least 18 years old, that is, that there exists a birth certificate, tied to the identify of the prover and digitally signed by a trusted certification authority, stating a birthdate consistent with the age claim.
	A ZKP allows this, without the prover having to reveal the birthdate.


	This document describes important aspects of the current state of the art in ZKP security, implementation, and applications.
	There are several use-cases and applications where ZKPs can add value.
	To better assess this it is useful to benchmark implementations under several metrics, evaluate tradeoffs between security and efficiency, and develop an interoperability basis.
    The security of a proof system is paramount for the system users, but efficiency is also essential for user experience.


%%%%%%%%%%%%%%%%%%%%%%%%%%%%%%%%%%%%%%%%%%%%%%%%%%%%%%%%%
%%%%%%%%%%%% ABOUT CHAPTER "SECURITY"
	The ``Security'' chapter introduces the theory and terminology of ZKP systems.
	A ZKP system can be described with three components: {\texttt setup}, {\texttt prove}, {\texttt verify}. 
	The {\texttt setup}, which can be implemented with various techniques, determines the initial state of the prover and the verifier, including private and common elements.
	The {\texttt prove} and {\texttt verify} components are the algorithms followed by the prover and verifier, respectively, possibly in an interactive manner.
	These algorithms are defined so as to ensure three main security requirements: completeness, soundness, and zero-knowledge.

	Completeness requires that if both {\texttt prove} and {\texttt verify} are correct, and if the statement is true, then at the end of the interaction the prover is convinced of this fact.
	Soundness requires that not even a malicious prover can convince the verifier of a false statement.
	Zero knowledge requires that even a malicious verifier cannot extract any information beyond the truthfulness of the given statement.  


%%%%%%%%%%%%%%%%%%%%%%%%%%%%%%%%%%%%%%%%%%%%%%%%%%%%%%%%%
%%%%%%%%%%%% ABOUT CHAPTER "IMPLEMENTATION"
	The ``Implementation'' chapter focuses on devising a framework for the implementation of ZKPs, which is important for interoperability.
	One important aspect to consider upfront is the representation of statements.
	In a ZKP protocol, the statement needs to be converted into a mathematical object.
	For example, in the case of proving that an age is at least 18, the statement is equivalent to proving that the private birthdate {\texttt Y$_1$-M$_1$-D$_1$} (year-month-day) satisfies a relation with the present date {\texttt Y$_2$-M$_2$-D$_2$}, namely that their distance is greater than or equal to 18 years.
	This simple example can be represented as a disjunction of conditions: 
{\tt Y$_2>$Y$_1$+18}, 
or {\tt Y$_2$=Y$_1$+18}~$\wedge$~{\tt M$_2$}$>${\tt M$_1$},
or {\tt Y$_2$=Y$_1$+18 }$\wedge${ \tt M$_2$=M$_1$ }$\wedge${ \tt D$_2$}$\geq${\tt D$_1$}.
	An actual conversion suitable for ZKPs, namely for more complex statements, can pose an implementation challenge. 
    There are nonetheless various techniques that enable converting a statement into a mathematical object, such as a circuit.
    This document gives special attention to representations based on a Rank-1 constraint system (R1CS) and quadratic arithmetic programs (QAP), which are adopted by several ZKP solutions in use today.
	Also, the document gives special emphasis to implementations of non-interactive proof systems.
\loosen


%%%%%%%%%%%%%%%%%%%%%%%%%%%%%%%%%%%%%%%%%%%%%%%%%%%%%%%%%
%%%%%%%%%%%% ABOUT CHAPTER "APPLICATIONS"
	The privacy enhancement offered by ZKPs can be applied to a wide range of scenarios.  
	The ``Applications'' chapter presents three use-cases that can benefit from ZKP systems: identity framework; asset transfer; regulation compliance. 
	In a privacy-preserving identity framework, one can for example prove useful personal attributes, such as age and state of residency, without revealing more detailed personal data such as birthdate and address.
	In an asset-transfer setting, financial institutions that facilitate transactions usually require knowing the identities of the sender and receiver, and the asset type and amount. 
	ZKP systems enable a privacy-preserving variant where the transaction is performed between anonymous parties, while at the same time ensuring they and their assets satisfy regulatory requirements.
	In a regulation compliance setting, ZKPs enables an auditor to obtain proof that a process satisfies a number of requirements, without having to learn details about how they were achieved.
	These use cases, as well as a wide range of many other conceivable privacy-preserving applications, can be enabled by a common set of tools, or gadgets, for example including commitments, signatures, encryption and circuits.
\loosen


%%%%%%%%%%%%%%%%%%%%%%%%%%%%%%%%%%%%%%%%%%%%%%%%%%%%%%%%%
%%%%%%%%%%%% CONCLUDING PARAGRAPH
	The interplay between security concepts and implementation guidelines must be balanced in %to enable 
the development of secure, practical, and interoperable ZKP applications.
	Solutions provided by ZKP technology must be ensured by careful security practices and realistic assumptions.
	This document aims to summarize security properties and implementation techniques that help achieve these goals.
		