%%% This file: defines commands and counters useful for cross-referencing feedback notes and corresponding changes

\newcommand{\ignoreDiff}[1]{#1}  % To mark areas to be ignored by the diff operation
\ifbool{boolDiff}{\setbool{boolShowFeedback}{true}}{}
\ifbool{boolShowFeedback}{}{\newcommandtwoopt{\revblock}[3][][]{}} 


%%%%%%%%%%%%%%%%%%%%%%%%%%%%
%%% COMMANDS FOR LEAVING PDF COMMENTS (e.g., POPUP ANNOTATIONS)

%%% Editorial note: All pop-up commands (introduced in version 0.1 as part of porting 
  % content to LaTeX) were disabled when finalizing version draft 0.2; will later 
	% address/collect these notes as suggestions for an upcoming revised version

\newcommand{\popupColor}[3][blue]{\pdfcomment[color=#1,opacity=0.5,author={#2}]{#3}}
\renewcommand{\popupColor}[3][blue]{}  %%% Disabling popup annotations, rendering ineffective some commands below
\newcommand{\suggest}[2][]{\popupColor[blue]{#1}{Suggestion: #2}}
\newcommand{\edited}[2][]{\popupColor[green]{#1}{Edited: #2}}
\newcommand{\smttodo}[2][]{\popupColor[red]{#1}{To-do: #2}}

%%%%%%%%%%%%%%%%%%%%%%%%%%%%
%%%%% MARK EDIT/REVISION BLOCKS
\def\letterEditBlock{E}   % E for edit; could also have been R for revision
\newcounter{cntRev}\renewcommand{\thecntRev}{\letterEditBlock\arabic{cntRev}}
\providecommandtwoopt{\revblock}[3][][0pt]{\marginnote{\hspace*{#2}\resizebox{.85in}{!}{{\normalsize\normalfont\color[rgb]{1,0,0}%
			\begin{minipage}{1.1in}\vspace{0em}\refstepcounter{cntRev}%
				\ifNotEmpty{#1}{\label{#1}} %adds a labels to the review edit
				\thecntRev\ifNotEmptyPrepend{#3}{: }
			\end{minipage}}}}}
\providecommandtwoopt{\revblockmini}[3][][]{{\scriptsize}}
%\renewcommand{\revblock}[2][]{}   % Uncomment this line to get ride of all margin notes
%\newcommand{\revblockRight}[3][]{}  %%% to delete in favor of using \revblock with the second optional argument


%%%%%%%%%%%%%%%%%%%%%%%%%%%%
%%%%%% Colors
\providecommand{\blue}[1]{{\color[rgb]{0,0,1}#1}}
\providecommand{\red}[1]{{\color[rgb]{1,0,0}#1}}
\providecommand{\dgreen}[1]{{\color[rgb]{0,.33,0}#1}}
\definecolor{mycoral}{rgb}{.9,.333,.2000}
\definecolor{brown}{rgb}{0.6,0.3,0}
\definecolor{LLgray}{rgb}{0.9,0.9,0.9}
\definecolor{LLyellow}{rgb}{1,1,.8}
\definecolor{darkgreen}{rgb}{0,0.5,0}


%%%%%%%%%%%%%%%%%%%%%%%%%%%%
%%%%%% Abbreviation commands that show some color-highlighted header within an item
%%% \Note, \Chan, \NoChan, etc. are useful prefix commands for the descriptive comments in the column ``Context and Changes''
\providesetbool{boolNeedNewLineBeforeChan}{false} %boolNeedNewLineBeforeChan: true after using \Note; becomes false when using \rowendL
\newcommand{\Note}{\ifbool{boolNeedNewLineBeforeChan}{\newline}{}{\bfseries\textcolor{blue}{-- Note: }}\setbool{boolNeedNewLineBeforeChan}{true}}
\newcommand{\Chan}{\ifbool{boolNeedNewLineBeforeChan}{\newline}{}{\bfseries\textcolor{mycoral}{-- Changed: }}\setbool{boolNeedNewLineBeforeChan}{true}}
\newcommand{\NoChan}{\Chan\ No change.}
\newcommand{\propContrib}{\ifbool{boolNeedNewLineBeforeChan}{\newline}{}{\bfseries\textcolor{blue}{-- Proposed contribution:} }}
\newcommand{\contributors}{\ifbool{boolNeedNewLineBeforeChan}{\newline}{}{\bfseries\textcolor{darkgreen}{-- Contributors:} }\setbool{boolNeedNewLineBeforeChan}{true}}
\newcommand{\submit}{\newline{\bfseries\textcolor{brown}{-- Submission mode:} }}
\newcommand{\ccontext}{{\ifbool{boolNeedNewLineBeforeChan}{\newline}{}\bfseries\textcolor{blue}{-- Context:} }\setbool{boolNeedNewLineBeforeChan}{true}}
\newcommand{\needscontributors}{\red{needs contributors}}


%%%%%%%%%%%%%%%%%%%%%%%%%%%%

\newcommand{\newcol}{&}
\newcommand{\rowendL}{\end{minipage}\\\hline}  %%%NEED TO COORDINATE WITH DEFINITION OF LAST COLUMN, currently using a minipage

\def\colNamePropContrib{Contribution topic}
\def\letterIndexPropContrib{C}
\def\rowIssueName{\emph{short description}}

\newcounter{cntItA} %counts the items sequentially: 1, 2, ..., 210
\newcounter{cntContrib}\renewcommand{\thecntContrib}{\letterIndexPropContrib\arabic{cntContrib}} %Counts the contribution topic: D1, D2, ..., D7 ...
\newcounter{cntItB}[cntContrib] %counts the item inside the Contribution: D1.1, D1.2, D1.3, D2.1, D2.2, .., D7.1, D7.2, ...
\renewcommand{\thecntItB}{\thecntContrib.\arabic{cntItB}}


% \inc... % increment some counter(s)
\newcommandtwoopt{\incItem}[2][][]{%
	\refstepcounter{cntItA}\ifNotEmpty{#1}{\label{it:#1}}\thecntItA \newcol 
	\refstepcounter{cntItB}\ifNotEmpty{#1}{\label{#1}}
	\ifNotEmptyElse{#2}{\def\itemPdfBkmExt{: #2}}{\def\itemPdfBkmExt{}}
	\pdfbookmark[3]{\thecntItB\itemPdfBkmExt}{pdfbkm:\thecntItB}\thecntItB \newcol } %incremment comment number



%\newcommand{\tabhead}{\rowColorHeader\# & Ref & \centering \scalebox{.75}{\mycol{Location in\\initial draft}} & \centering Comment received & \scalebox{.75}{\mycol{Related\\items}} & \centering Reply notes & \begmin{.04}\scalebox{.8}{\mycol{Revie-\\wed?}}\myendmini}
\setlength{\tabcolsep}{1ex}


%Use of commands defined with optional arguments inside cells can break the workings of tabular
%See complicated solution in https://tex.stackexchange.com/questions/277140/optional-argument-breaks-command-with-multicolumn
%As a temporary solution, I'll just define all commands as mandatory
%\newcommand{\headreviewer}[2]{\rowcolor{LLyellow}\incIssue[#1]\pdfbookmark[2]{\thecntContrib\ --- #2}{pdfbkm:\thecntContrib}\textbf{\thecntContrib} & Location & Comments from #2 & & & \rowend}


%%% 
%\newcommandtwoopt[2][.320][.320]{\beginlongtabIssue}{%
\newcommand{\beginlongtabIssue}{%
\begin{longtable}{%
|>{\begmin{.025}\centering}c<{\myendmini}%  #
|>{\begmin{.035}\centering}c<{\myendmini}%  Ref
|>{\begmin{.080}}c<{\myendmini}%            Old location
|>{\begmin{.320}}c<{\myendmini}%              Comment / Issue
|>{\begmin{.060}\centering}c<{\myendmini}%  Related
|>{\begmin{.320}}c<{\myendmini}%              Reply notes / Notes about changes
|>{\begmin{.050}\centering}c<{}|%           Rev [in main document]
% The last column needs to be ended explicitly with \rowend, which contains a \end{minipage} followed by \\
}%
}



%%%%%%%%%%%%%%%%%%%%%%%%%%%%%%%%%%%%%%%%%%%%%%%%%%%%%%%%%%%%%%%%%%%%%%%%%%%%%%%%
\newcommand{\headerForIssue}{%
	  \centering \#
	& \centering \scalebox{.7}{Item id}
	& \centering Location %Old location
	& \centering \textbf{\colNamePropContrib\ \thecntContrib}: \rowIssueName{}
	& \centering Related %\scalebox{.75}{\mycol{Related\\items}}
	& \centering Incorporated changes
	& \centering Edit id
}


\newcommand{\githubissue}[1]{\href{https://github.com/zkpstandard/zkreference/issues/#1}{GI#1}}


%%%%%%%%%%%%%%%%%%%%%%%%%%%%%%%%%%%%%%%%%%%%%%%%%%%%%%%%%%%%%%%%
%%%%%%%%%%% TO GENERATE A LIST OF COMMENTS/CONTRIBUTIONS

%creates the file name zkproof-community-reference.loc with the index of contribution issues/items
\newcommand\listcontributionname{List of Contributions}
\newlistof{contribution}{loc}{\listcontributionname}  %%%[section]
\newlistof{contributions}{loc2}{\listcontributionname}  % Duplicate on purpose



%%%%%%%%%%%%%%%%%%%%%%%%%%%%%%%%
%%% Each `issue' is a `longtable' (which can break pages) of comment/contribution items. 
%%% Each `issue' starts with \newIssue and ends with \myendIssue
%%% Besides the automatic header, each row of the table will be an `item' introduced by command \incItem

\newcommand{\newIssue}[2]{%
	\refstepcounter{cntContrib}\label{#1}
	\addcontentsline{loc}{contribution}{\thecntContrib: #2}  %%% \colNamePropContrib\ 
	\addcontentsline{loc2}{contributions}{\thecntContrib: #2}
	\renewcommand{\rowIssueName}{#2}
	\noindent\beginlongtabIssue\hline%
			\rowcolor{LLyellow} %must be the first command in the cell, namely before pdfbookmark
			\pdfbookmark[2]{\thecntContrib: \rowIssueName{}}{pdfbkm:\thecntContrib}% adds a pdf bookmark if this is the first header for this Issue	
			\headerForIssue\rowend\hline\hline%
		\endfirsthead\hline%
			\rowcolor{LLgray}\headerForIssue\rowend\hline\hline%
		\endhead%ensures that all lines above this one will repeat as a header
}

\newcommand{\myendIssue}{\end{longtable}}
