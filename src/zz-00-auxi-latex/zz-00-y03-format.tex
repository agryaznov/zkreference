%%%%%%%%%%%%%%%%%%%%%%%%%%%%%%%%%%%%%%%%%%%%%%%%%%%%%%%%%%%%%%%%
%%%%%%%%%%%%%%%%%%%% Related to package titlesec

\def\fontfamheads{\sffamily}  %% font family type for headings: sans serif, serif, ... ... check the options when loading titlesec

\titlespacing*{\chapter}{0pt}{-25pt}{2em}
\titleformat{\chapter}[display]{\fontfamheads\huge\centering}{Chapter \thechapter.\enskip}{20pt}{\Huge}
\titleformat{\chapter}[hang]{\fontfamheads\huge\bfseries}{\chaptertitlename\ \thechapter.}{.5em}{} 
\titlespacing{\subsubsection}{0pt}{1em}{0em}

%%% Same font size as section header ?
\renewcommand{\cfttoctitlefont}{\fontfamheads\bfseries\Large}
\renewcommand{\cftloftitlefont}{\fontfamheads\bfseries\Large}
\renewcommand{\cftlottitlefont}{\fontfamheads\bfseries\Large}

\renewcommand\cftchapafterpnum{\vskip0pt} %spacing between items in toc
\renewcommand\cftsecafterpnum{\vskip0pt} %spacing between items in toc


%%%%%%%%%%%%%%%%%%%%%%%%%%%%%%%%%%%%%%%%%%%%%%%%%%%%%%%%%%%%%%%%
%%%%%%%%%%%%%%%%%%%% \presection: 
% prints a centered and unnumbered section header, and adds the header to the table of contents
% 2nd optional argument has default value `section'
% e.g., \presection[short-section-name]{my-very-long-section-name}
% e.g., \presection[short-subsection-name][subsection]{my-very-long-subsection-name}

\newcommandtwoopt{\presection}[3][][section]{\phantomsection\section*{\centering #3}\ifEmptyElse{#1}{\addcontentsline{toc}{#2}{#3}}{\addcontentsline{toc}{#2}{#1}}}


%%%%%%%%%%%%%%%%%%%%%%%%%%%%%%%%%%%%%%%%%%%%%%%%%%%%%%%%%%%%%%%%
%%%%%%%%%%%% Vertical Spacing

\setitemize{itemsep=0em}  % requires \usepackage{enumitem}
\setlist[itemize]{itemsep=.5ex, topsep=0pt}
\setlist[enumerate]{itemsep=.5ex, topsep=0pt}

\setlength{\parskip}{\medskipamount} 
\setlength{\parskip}{.75em}
\setlength{\parindent}{0pt}

%% Changing the paragraph command to automatically end with a dot, and so that it can include a label as an optional argument  
\let\originalparagraph\paragraph
\renewcommand{\paragraph}[2][.]{\vskip0pt\vspace{-.5em}\originalparagraph{#2#1}}
\renewcommand{\paragraph}[2][.]{{\fontfamheads\vskip.5em\noindent\textbf{#2#1}~}}  %% similar to \paragraph, but without the extra vertical spacing

\newcommand{\mypar}[1]{\paragraph{#1}\mbox{}\\\vspace{-1ex}}  %%% paragraph whose label ends the line

%%% \inpar: similar to a paragraph command, but without any extra vertical spacing
\newcommand{\inpar}[2][.]{{\fontfamheads\vskip0pt\noindent\textbf{#2#1}~}}  %% similar to \paragraph, but without the extra vertical spacing

% \parhead: originally defined in package dtrt
\providecommand{\parhead}[1]{\smallskip \noindent {\bfseries\boldmath\ignorespaces \dt@MaybeAddPunct{#1}}\hskip 0.9em plus 0.3em minus 0.3em \dt@ignorespacesandimplicitepars}

\newcommand{\question}[1]{\subsection{#1}}
\newcommand{\subquestion}[1]{\parhead{#1}}
\newcommand{\subanswer}[1]{\parhead{#1}}


%%%%%%%%%%%%%%%%%%%%%%%%%%%%%%%%%%%%%%%%%%%%%%%%%%%%%%%%%%%%%%%%
%%%%%%%%%%%% Bookmarks and meta-bookmarks

\setcounter{tocdepth}{3} % depth at which to show contents in the table of contents % needs package tocloft
\setcounter{secnumdepth}{3} % to show numbering for subsubsections

\hypersetup{
	linktocpage,
	bookmarksnumbered,bookmarksopen,bookmarksdepth=3,bookmarksopenlevel=1,
	pdfdisplaydoctitle=true,
	pdfstartview={FitH},    % fits the width of the page to the window
	breaklinks=true,
	colorlinks=true,allcolors=darkblue
}


%%%%%%%%%%%%%%%%%%%%%%%%%%%%%%%%%%%%%%%%%%%%%%%%%%%%%%%%%%%%%%%%
%%%%%%%%%%%%%%%%%%%% Related to headers, footers and the configuration related to package fancyhdr
%use \ifoddeven to define different headers between odd and even pages

\renewcommand{\headrulewidth}{0pt}
\setlength{\headheight}{13.6pt}

%\renewcommand{\chaptermark}[1]{\markboth{#1}{}}
%\renewcommand{\sectionmark}[1]{\markright{\thesection\ #1}}

\renewcommand{\chaptermark}[1]{\markboth{\chaptername~\thechapter: #1}{}}
\renewcommand{\sectionmark}[1]{\markright{\thesection~ #1}}


\fancypagestyle{cnts}{\fancyhead{}
	\lhead{\ifoddeven{\zkpcomrefabbrev~\zkpcomrefversion}{}}  % . \nouppercase{\leftmark}
	\rhead{\ifoddeven{}{Contents}}  % \nouppercase{\rightmark}
	\cfoot{\thepage}
}

\fancypagestyle{exec-summ}{\fancyhead{}
	\lhead{\ifoddeven{\zkpcomrefabbrev~\zkpcomrefversion}{}}
	\rhead{\ifoddeven{}{Executive summary}}
	\cfoot{\thepage}
}

\fancypagestyle{mainmatter}{\fancyhead{}
	\lhead{\ifoddeven{\zkpcomrefabbrev~\zkpcomrefversion. \nouppercase{\leftmark}}{}}
	\rhead{\ifoddeven{}{Section \nouppercase{\rightmark}}}
	\cfoot{\thepage}
}

\fancypagestyle{acks}{\fancyhead{}
	\lhead{\ifoddeven{Acknowledgments}{}}
    \rhead{\ifoddeven{}{Acknowledgments}}
	\cfoot{\thepage}}

\fancypagestyle{references}{\fancyhead{}
	\lhead{\ifoddeven{References}{}}
	\rhead{\ifoddeven{}{References}}
	\cfoot{\thepage}}

\fancypagestyle{appendix}{\fancyhead{}
	\lhead{\ifoddeven{\zkpcomrefabbrev~\zkpcomrefversion. Appendix \nouppercase{\leftmark}}{}}
	\rhead{\ifoddeven{}{Section \nouppercase{\rightmark}}}
	\cfoot{\thepage}
}

\def\intentionallyblank{\scalebox{1.5}{\color[rgb]{0.85,0.85,0.85}Page intentionally blank}}
\def\raisedintblank{\smash{\raisebox{-3.5em}{\intentionallyblank}}}
\fancypagestyle{intentionallyblank}{\fancyhead{}\chead{\smash{\raisebox{-3.5em}{\raisedintblank}}}\cfoot{}}

\fancypagestyle{blankpagewithnumber}{\fancyhead{}\chead{\raisedintblank}\cfoot{\thepage}} %to avoid

\newcommand\cleartooddpage{\clearpage\ifodd\value{page}\else
	%avoid the \rightmark of the `mainmatter' page style, but still include a page number
	\null\thispagestyle{blankpagewithnumber}\clearpage\fi}

\fancypagestyle{empty}{\fancyhead{}\cfoot{}}


%%%%%%%%%%%%%%%%%%%%%%%%%%%%%%%%%%%%%%%%%%%%%%%%%%%%%%%%%%%%%%%%
%%%%%%%%%%%% Table of Contents and References

\makeatletter
\@ifpackageloaded{babel}{%
	%if using babel
	\addto\captionsenglish{% Replace "english" with the language you use %is using package `babel', must specify the language
		\renewcommand{\contentsname}{\phantomsection\pdfbookmark[2]{Table of Contents}{pdfbkm:toc}\label{toc:toc}Table of Contents}
		\renewcommand{\bibname}{\pdfbookmark[1]{References}{pdfbkm:refs}References}
		}%
		%%% \renewcommand{\refname}
	}%
	{%
		\renewcommand{\contentsname}{\phantomsection\pdfbookmark[2]{Table of Contents}{pdfbkm:toc}\label{toc:toc}Table of Contents}%
		\renewcommand{\bibname}{\pdfbookmark[1]{References}{pdfbkm:refs}References}
	}
\makeatother



%%%%%%%%%%%%%%%%%%%%%%%%%%%%%%%%%%%%%%%%%%%%%%%%%%%%%%%%%%%%%%%%
%%%%%% FORMATTING of others

%%% \topscite: citation to be used in places 
%\newcommand{\topscite}[1]{\texorpdfstring{\cite{#1}}{{\protect\NoHyper\cite{#1}\protect\endNoHyper}}}

%\fboxmp: add an fbox, but the parameter will also be wrapped within a minipage. The optional arg is the width factor
\newcommand{\fboxmp}[2][1]{\fbox{\begin{minipage}{#1\textwidth-2\fboxrule-2\fboxsep}#2\end{minipage}}}

\newcommand\twodigits[1]{\ifnum#1<10 0#1\else #1\fi}
\def\todayext{\ifcase\month\or
  January\or February\or March\or April\or May\or June\or
  July\or August\or September\or October\or November\or December\fi
  \space\twodigits{\number\day}, \number\year}


%%%%%%%%%%%%%%%%%%%%%%%%%%%%%%%%%%%%%%%%%%%%%%%%%%%%%%%%%%%%%%%%
%%%%%% Formatting related to bibliography


%%% Options for biblatex ... could also be called when loading the package
\ExecuteBibliographyOptions{sorting=anyt, % anyt: sorts by alphabetic label, name, year, title
	backref=true, % Search \renewbibmacro*{pageref} for the styling of back page references
	bibencoding=utf8,%
	dateabbrev=false,%
	giveninits=true,%
	isbn=false,url=false,%
	maxnames=10, %minnames=4, %    %maxnames=99
	minbibnames=4,maxbibnames=99,
	minalphanames=5,maxalphanames=8,
	mincitenames=4,maxcitenames=10,
	}


\DefineBibliographyStrings{english}{%
backrefpage = {Cited on p.}, %
backrefpages = {Cited on pp.} %
}

\DeclareListFormat{location}{}

%%% LB: code to avoid indentation of lines after the first
\setlength{\bibhang}{0em}
\defbibenvironment{bibliography}%
{\list{\printtext[labelalphawidth]{%
        %\printfield{prefixnumber}%
        \textbf{\printfield{labelalpha}}}} %\list
{\setlength{\leftmargin}{\bibhang}%
\setlength{\itemindent}{2.5em}%  %-\leftmargin
\setlength{\itemsep}{\bibitemsep}%
\setlength{\parsep}{\bibparsep}}%
%\renewcommand*{\makelabel}[1]{\hss##1}%
}
{\endlist} %\endlist
{\item} %%%\makelabel{#1}

%\patchcmd requires package etoolbox
\patchcmd{\thebibliography}{\chapter*}{\section*}{}{} %will prevent the bibliography from breaking the page


%%%%%%%%%%%%%%%%%%%%%%%%%%%%%%%%%%%%%%%%%%%%%%%%%%%%%%%%%%%%%%%%
%%%%%% Microtyping to improve paragraph ending

%%% \loosen (requires \usepackage{microtype}): uses microtype features to help prevent some line breaks in paragraphs. Does not currently work with xetex
\def\loosen{\looseness=-1}

%%% Luis B: command \xetexfontscale (as alternative to \loosen) for use with xetex, where the microtype package does not provide kerning or tracking, 
\ifPDFTeX\newcommand{\xetexfontscale}[2]{#2\loosen}
\else
\usepackage{anyfontsize,xfp}
\makeatletter
\newcommand{\xetexfontscale}[2]{\ifxetex\makeatother%\normalsize% Capture font definitions for \normalsize
\xdef\f@size@normalsize{\f@size}%
\xdef\f@baselineskip@normalsize{\f@baselineskip}%
\fontsize{\fpeval{(\f@size@normalsize)*#1}}{\fpeval{(\f@baselineskip@normalsize)*#1}}\selectfont{}#2\fontsize{\fpeval{(\f@size@normalsize)}}{\fpeval{(\f@baselineskip@normalsize)}}\selectfont{}\else{}#2\fi}
\makeatother
\fi

\renewcommand\linenumberfont{\normalfont\tiny\rmfamily}
