%%%%%%%%%%%%%%%%%%%%%%%%%%%%%%%%%%%%%%%%%%%%%%%%%%%%%%%%%%%%%%%%%%%%%%%%%%%%%%%%
\chapter*{Acknowledgments}\addcontentsline{toc}{chapter}{Acknowledgments}
\label{app:acknowledgments}

The development of this community reference counts with the support of numerous individuals.


%%%%%%%%%%%%%%%%%%%%%%%%%%%%%%%%%%%%%%%%%%%%%%%%%%%%%%%%%%%%
\inpar{Version 0}
The ``proceedings'' of the 1st ZKProof workshop (Boston, May 2018) formed the initial basis for this document.
The contributions were organized in three tracks:

\begin{itemize}\setlength{\itemsep}{1ex}
	
\item \textbf{Implementation track.} 
	Chairs: Sean Bowe, Kobi Gurkan, Eran Tromer.
	Participants: Benedikt Bünz, Konstantinos Chalkias, Daniel Genkin, Jack Grigg, Daira Hopwood, Jason Law, Andrew Poelstra, abhi shelat, Muthu Venkita\-subramaniam, Madars Virza, Riad S.\ Wahby, Pieter Wuille.
	
\item \textbf{Applications Track.}
	Chairs: Daniel Benarroch, Ran Canetti, Andrew Miller.
	Participants: Shashank Agrawal, Tony Arcieri, Vipin Bharathan, Josh Cincinnati, Joshua Daniel,  Anuj Das Gupta, Angelo De Caro, Michael Dixon, Maria Dubovitskaya, Nathan George, Brett Hemenway Falk, Hugo Krawczyk, Jason Law, Anna Lysyanskaya, Zaki Manian, Eduardo Morais, Neha Narula, Gavin Pacini, Jonathan Rouach, Kartheek Solipuram, Mayank Varia, Douglas Wikstrom, Aviv Zohar.

\item \textbf{Security track.}
	Chairs: Jens Groth, Yael Kalai, Muthu Venkitasubramaniam.
	Participants: Nir Bitansky, Ran Canetti, Henry Corrigan-Gibbs, Shafi Goldwasser, Charanjit Jutla, Yuval Ishai, Rafail Ostrovsky, Omer Paneth, Tal Rabin, Maryana Raykova, Ron Rothblum, Alessandra Scafuro, Eran Tromer, Douglas Wikström.

\end{itemize}


%%%%%%%%%%%%%%%%%%%%%%%%%%%%%%%%%%%%%%%%%%%%%%%%%%%%%%%%%%%%
\paragraph{Version 0.1}
Prior to the 2nd ZKProof workshop, the ZKProof organization team requested feedback from NIST about the developed documentation.
The NIST PEC team (Luís Brandão, René Peralta, Angela Robinson) then elaborated the ``NIST comments on the initial ZKProof documentation'' \cite{2019:PEC:zkproof-comments} with 28 comments/suggestions for subsequent development of a ``Community Reference Document''.
Luís Brandão ported to LaTeX the proceedings into a LaTeX version, along with inline comments, which became named as version 0.1.


%%%%%%%%%%%%%%%%%%%%%%%%%%%%%%%%%%%%%%%%%%%%%%%%%%%%%%%%%%%%
\paragraph{Version 0.2}
The contributions from version 0.1 to version 0.2 followed the editorial process initiated at the 2nd ZKProof Workshop (Berkeley, April 2019).
Several suggested contributions stemmed from the breakout discussions in the workshop, which were possible by the collaboration of \emph{scribes}, \emph{moderators} and \emph{participants}, as documented in the Workshop Notes \cite{2019:zkproof:notes-2nd-workshop}.
The actual content contributions were developed thereafter by several \emph{contributors}, including Yu Hang, Eduardo Morais, Justin Thaler, Ivan Visconti, Riad Wahby and Yupeng Zhang, besides the NIST PEC team (Luís Brandão, René Peralta, Angela Robinson) and the Editors team (Daniel Benarroch, Luís Brandão, Eran Tromer).
The detailed description of the changes, contributions and contributors appears in the ``diff'' version of the community reference.


%%%%%%%%%%%%%%%%%%%%%%%%%%%%%%%%%%%%%%%%%%%%%%%%%%%%%%%%%%%%
\paragraph{Version 0.3}
The main update was the revision of the \hyperref[chap:paradigms]{paradigms} chapter.
The old section ``Taxonomy of Constructions'' was replaced by a new section with  ``\hyperref[paradigms:background]{background}'' and then two sections on ``\hyperref[paradigms:IT]{IT Proof Systems}'' and ``\hyperref[paradigms:CC]{Cryptographic Compilers}'' were added.
This change was inspired by, and mostly adapted from, two blogposts titled ``Zero-Knowledge Proofs from Information-Theoretic Proof Systems'' (parts 1 and 2) by Yuval Ishai \cite{2020:Ish:zkproof-blog:ZKPs-from-IT-proof-systems}.
The content in the \nameref{paradigms:IT:linear-IOP:IP-based} (under \nameref{paradigms:IT:linear-IOP}) subsection was provided by Justin Thaler.
The revised chapter was structured and edited by the ZKProof Editors team (Daniel Benarroch, Luís Brandão, Mary Maller, Eran Tromer). 
Some editorial corrections were externally suggested, including during the public review phase, as described in the ``Call for contributions to the ZKProof Community Reference --- Review cycle 2020: from version 0.2 to 0.3.'' \cite{2020:zkproof:call-contribs}.
Suggestions from Jens Groth improved \reftab{tab:example-scenarios-zkps} in \refsec{security:terminology}. 
Thanks for several other comments from various readers.
The integration of some received suggestions is left for future versions.


%%%%%%%%%%%%%%%%%%%%%%%%%%%%%%%%%%%%%%%%%%%%%%%%%%%%%%%%%%%%
\paragraph{Miscellaneous}
A general ``thank you'' goes to all who have so far collaborated with the ZKProof initiative.
This includes the workshop speakers, participants, organizers and sponsors, as well as the ZKProof steering committee and program committee members, and the participants in the online ZKProof forum.
Detailed information about ZKProof is available on the \href{https://zkproof.org/}{zkproof.org} website.
