%%%%%%%%%%%%%%%%%%%%%%%%%%%%%%%%%%%%%%%%%%%%%%%%%%%%%%%%%%%%%%%%%%%%%%%%%%%%%%%%
\section{Gadgets within predicates}
\label{apps:gadgets-within-predicates}

Formalizing the security of these protocols is a very difficult task, especially since there is no predetermined set of requirements, making it an ad-hoc process.

Use-cases must be sure to distinguish between privacy requirements and security guarantees.
We discuss the use-case case of privacy-preserving asset transfer to illustrate the difference.

Secure asset transfer is possible at several financial institutions, provided that the institution has knowledge of the identities of the sender, recipient, asset, and amount.
In a privacy-preserving asset transfer, the identities of sender and recipient may be concealed even from the entity administering the transfer.
It is important to note that a successful transfer must meet privacy requirements as well as provide security guarantees.

Privacy requirements might include the anonymity of sender and recipient, concealment of asset type and asset amount.
Security guarantees might include the inability of anyone besides the sender to initiate a transfer on the sender's behalf or the inability of a sender to execute a transfer of asset type without sufficient holdings of the asset. 

Here we outline a set of initial gadgets to be taken into account.
See \reftab{tab:list-gadgets} for a simple list of gadgets --- this list should be expanded continuously and on a case by case basis.
For each of the gadgets we write the following representations, specifying what is the secret / witness, what is public / statement:

NP statements for non-technical people:

\hfill\begin{minipage}{.92\textwidth}\bfseries
	For the [public] chess board configurations $A$ and $B$;\newline
	I know some [secret] sequence $S$ of chess moves;\newline
	such that when starting from configuration $A$, and applying $S$, all moves are legal and the final configuration is $B$.
\end{minipage}

General form (Camenisch-Stadler): \textbf{Zk \{ ( wit):   P(wit, statement)  \}}

\restoline{Example of ring signature: \textbf{Zk \{ (sig):  VerifySignature(P1, sig) or VerifySignature(P2, sig)  \}}}


\newcounter{cntGadget}\setcounter{cntGadget}{0}
\renewcommand{\thecntGadget}{G\arabic{cntGadget}}
\newcommand{\newGadget}[1][]{\refstepcounter{cntGadget}\protect\ifthenelse{\equal{#1}{}}{}{\label{#1}}\thecntGadget}


%%%%%%%%%%%%%%%%%%%%%%%%%%%%%%%%%%%%%%%%
\begin{table}[H]\centering
\CIL\mytabcap[tab:list-gadgets]{List of gadgets}{List of gadgets}
\vskip0pt\nolinenumbers\begin{edtable}{tabular}{|>{\begmin{.04}\centering}c<{\myendmini}|>{\begmin{.23}}l<{\myendmini}||>{\begmin{.48}}l<{\myendmini}|>{\begmin{.13}\centering}c<{}|}
\hline \rowcolor{colorRowHead} \bfseries \raisebox{-3ex}{\#}
	& \centering \bfseries \raisebox{-3ex}{Gadget name}
	& \centering \bfseries Intuitive description of the initial gadget (before adding ZKP)
	& \centering \bfseries Table with examples \rowend
\hline 
\hline 
\newGadget[gad:commitment] 
    & Commitment 
	& Envelope
    & \reftab{tab:gadget-commitment-envelope} \rowend
\hline
\newGadget[gad:signatures] 
    & Signatures 
	& Signature authorization letter 
    & \reftab{tab:gadget-signature} \rowend
\hline\newGadget[gad:encryption] 
    & Encryption 
	& Envelope with a receiver stamp 
    & \reftab{tab:gadget-encryption} \rowend
\hline
\newGadget[gad:dist-decryption] 
    & Distributed decryption 
	& Envelope with a receiver stamp that requires multiple people to open 
    & \reftab{tab:gadget-dist-decryption} \rowend
\hline
\newGadget[gad:rand-func] 
    & Random function 
	& Lottery machine 
    & \reftab{tab:gadget-random-function} \rowend
\hline
\newGadget[gad:set-membership] 
    & Set membership 
	& Whitelist/blacklist 
    & \reftab{tab:gadget-set-membership} \rowend
\hline
\newGadget[gad:mix-net] 
    & Mix-net 
	& Ballot box 
    & \reftab{tab:gadget-mix-net} \rowend
\hline
\newGadget[gad:gen-calculations] 
    & Generic circuits, TMs, or RAM programs 
	& General calculations 
    & \reftab{tab:gadget-general-computation} \rowend
\hline
\end{edtable}\vspace{1em}
\end{table}


%%%%%%%%%%%%%%%%%%%%%%%%%%%%%%%%%%%%%%%%
\newcommand{\headerRowForGadgets}{\rowcolor{colorRowHead}Enhanced gadget (after adding ZKP) 
	& ZKP statement (in a PoK notation) 
	& Prover knows a witness ...
	& ...for the public instance ...
	& ...s.t. the following predicate holds
	%%%& Technical notation (API) 
	\rowend}

%the six arguments are factors for the width of the 6 columns
\newenvironment{gadgettabular}[5]{%
	\begin{center}\renewcommand{\thelinenumber}{}\footnotesize\begin{edtable}{tabular}{%
		|>{\begmin{#1}\setlinenumbermargin{1em}\internallinenumbers}l<{\myendmini}|%
		>{\begmin{#2}}l<{\myendmini}|%
		>{\begmin{#3}}l<{\myendmini}|%
		>{\begmin{#4}}l<{\myendmini}|%
		%%%>{\begmin{#5}}l<{\myendmini}|
		>{\begmin{#5}}c|
		}
		\hline \headerRowForGadgets \hline
	}{\end{edtable}\vspace{1em}\end{center}}


%%%%%%%%%%%%%%%%%%%%%%%%%%%%%%%%%%%%%%%%
\begin{table}[H]\centering
\CIL\mytabcap[tab:gadget-commitment-envelope]{Commitment gadget}{Commitment gadget (\ref{gad:commitment}; envelope)}
\vskip-.5em\nolinenumbers\begin{gadgettabular}{.26}{.16}{.17}{.12}{.18}  %make it sum to .89
      I know the value hidden inside this envelope, even though I cannot change it
	& Knowledge of committed value(s) (openings)
    & Opening $O = (v,r)$ containing a value and randomness
    & Commitment $C$ %%Committed value(s) $C$
    & $C = \Comm(v,r)$ %%%, com\-po\-nent-wise if there are multiple $C, O$
    \rowend
\hline 
      I know that the value hidden inside these two envelopes are equal
    & Equality of committed values
    & Openings $O_1 = (v, r_1)$ and $O_2 = (v, r_2)$
    & Commitments $C_1$ and $C_2$
    & $C_1 = \Comm(v,r_1)$ and $C_2 = \Comm(v,r_2)$
    \rowend
\hline
      I know that the values hidden inside these two envelopes are related in a specific way
    & Relationships between committed values -- logical, arithmetic, etc.
	& Openings $O_1 = (v_1, r_1)$ and $O_2 = (v_2, r_2)$
	& Commitments $C_1$ and $C_2$, relation $R$
	& $C_1 = \Comm(v_1,r_1)$, $C_2 = \Comm(v_2,r_2)$, and $R(v_1, v_2) = \texttt{True}$
		\rowend
\hline
      The value inside this envelope is within a particular range
	& Range proofs
	& Opening $O = (v,r)$
	& Commitment $C$, interval $I$
	& $C = \Comm(v,r)$ and $v$ is in the range $I$
		\rowend
\hline
\end{gadgettabular}
\end{table}


%%%%%%%%%%%%%%%%%%%%%%%%%%%%%%%%%%%%%%%%
\begin{table}[H]\centering
\CIL\mytabcap[tab:gadget-signature]{Signature gadget}{Signature gadget (\ref{gad:signatures}; signature authorization letter)}
\vskip-.5em\nolinenumbers\begin{gadgettabular}{.18}{.25}{.12}{.15}{.19}  %make it sum to .89
      Secret valid signature over commonly known message
	& Knowledge of a secret signature $\sigma$ on a commonly known message $M$
    & Signature $\sigma$
    & Verification key $VK$, message $M$
    & Verify$(VK, M, \sigma) = \tt True$
    \rowend
\hline 
      Secret valid signature over committed message
	& Knowledge of a secret signature $\sigma$ on a commonly known commitment $C$ of a secret message $M$
	& Opening $O$, signature $\sigma$
	& Verification key $VK$, commitment $C$
	& $C = Comm(M)$ and Verify$(VK, M, \sigma) = \tt True$
    \rowend
\hline
\end{gadgettabular}
\end{table}


%%%%%%%%%%%%%%%%%%%%%%%%%%%%%%%%%%%%%%%%
\begin{table}[H]\centering
\CIL\mytabcap[tab:gadget-encryption]{Encryption gadget}{Encryption gadget (\ref{gad:encryption}; envelope with a receiver stamp)}
\vskip-.5em\nolinenumbers\begin{gadgettabular}{.24}{.16}{.15}{.16}{.18}  %make it sum to .89
      The output plaintext(s) correspond to the public ciphertext(s).
    & Knowledge of a secret plaintext $M$
    & Secret decryption key $SK$
    & Ciphertext(s) $C$ and Encryption key $PK$
    & $Dec(SK, C) = M$, component-wise if $\exists$ multiple $C$ and $M$
    \rowend
\hline
\end{gadgettabular}
\end{table}


%%%%%%%%%%%%%%%%%%%%%%%%%%%%%%%%%%%%%%%%%%%%%%%%%%%%%%%%
\begin{table}[H]\centering
\CIL\mytabcap[tab:gadget-dist-decryption]{Distributed-decryption gadget}{Distributed-decryption gadget\vskip0pt(\ref{gad:dist-decryption}; envelope with a receiver stamp that requires multiple people to open)}
\vskip-.5em\nolinenumbers\begin{gadgettabular}{.21}{.16}{.165}{.125}{.23}  %make it sum to .89
      The output plaintext(s) correspond to the public ciphertext(s).
    & Knowledge of a secret plaintext $M$
	  & Secret shares $[SK_i]$ of the decryption key $SK$
    & Ciphertext(s) $C$ and Encryption key $PK$
	  & $SK = Derive([SK_i])$ and $Dec(SK, C) = M$, com\-po\-nent-wise if $\exists$ multiple $C$
   \rowend
\hline
\end{gadgettabular}
\end{table}


%%%%%%%%%%%%%%%%%%%%%%%%%%%%%%%%%%%%%%%%%%%%%%%%%%%%%%%%
\begin{table}[H]\centering
\CIL\mytabcap[tab:gadget-random-function]{Random-function gadget}{Random-function gadget (\ref{gad:rand-func}; lottery machine)}
\vskip-.5em\nolinenumbers\begin{gadgettabular}{.17}{.31}{.13}{.12}{.16}  %make it sum to .89
      Verifiable random function (VRF)
    & VRF was computed from a secret seed and a public (or secret) input
    & Secret seed $W$
    & Input $X$, Output $Y$
    & $Y = VRF(W, X)$
    \rowend
\hline
\end{gadgettabular}
\end{table}


%%%%%%%%%%%%%%%%%%%%%%%%%%%%%%%%%%%%%%%%%%%%%%%%%%%%%%%%
\begin{table}[H]\centering
\CIL\mytabcap[tab:gadget-set-membership]{Set-membership gadget}{Set-membership gadget (\ref{gad:set-membership}; whitelist/blacklist)}
\vskip-.5em\nolinenumbers\begin{gadgettabular}{.18}{.23}{.16}{.14}{.18}  %make it sum to .89
      Accumulator
    & Set inclusion
	& Secret element $X$
	& Public set $S$ 
	& $X \in S$ 
	\rowend
\hline
      Universal accumulator 
    & Set non-inclusion
	& Secret element $X$ 
	& Public set $S$ 
	& $X \notin S$ 
	\rowend
\hline
      Merkle Tree	
    & Element occupies a certain position within the vector
    & Secret element $X$ 
	& Public vector $V$ 
	& $X = V[i]$ for some $i$ 
	\rowend
\hline
\end{gadgettabular}
\end{table}


%%%%%%%%%%%%%%%%%%%%%%%%%%%%%%%%%%%%%%%%
\begin{table}[H]\centering
\CIL\mytabcap[tab:gadget-mix-net]{Mix-net gadget}{Mix-net gadget (\ref{gad:mix-net}; ballot box)}
\vskip-.5em\nolinenumbers\begin{gadgettabular}{.165}{.275}{.12}{.155}{.175}  %make it sum to .89
      Shuffle
    & The set of plaintexts in the input and the output ciphertexts respectively are identical.
    & Permutation $\pi$, Decryption key $SK$
    & Input ciphertext list $C$ and Output ciphertext list $C'$
    & $\forall j, Dec(SK, C'_j) = Dec(SK, \pi(C_j))$
    \rowend
\hline
      Shuffle and reveal
	& The set of plaintexts in the input ciphertexts is identical to the set of plaintexts in the output.
	& Permutation $\pi$, Decryption key $SK$
	& Input ciphertext list C and Output plaintext list $P$
	& $\forall j, P_j = Dec(SK, \pi(C_j))$
	\rowend
\hline
\end{gadgettabular}
\end{table}


%%%%%%%%%%%%%%%%%%%%%%%%%%%%%%%%%%%%%%%%
\begin{table}[H]\centering
\CIL\mytabcap[tab:gadget-general-computation]{Generic-computation gadget}{Generic circuits, TMs, or RAM programs gadgets (\ref{gad:gen-calculations}; general calculations)}
\vskip-.5em\nolinenumbers\begin{gadgettabular}{.21}{.20}{.13}{.23}{.12}  %make it sum to .89
		 There exists some secret input that makes this calculation correct
    & ZK proof of correctness of circuit/Turing machine/RAM program computation
    & Secret input w
    & Program $C$ (either a circuit, TM, or RAM program), public input $x$, output $y$
    & $C(x, w) = y$
    \rowend
\hline
      This calculation is correct, given that I already know that some sub-calculation is correct
	& ZK proof of verification + post-processing of another output (Composition)
	& Secret input $w$
	& Program $C$ with subroutine $C'$, public input $x$, output $y$, intermediate value $z = C'(x, w)$, zk proof $\pi$ that $z = C'(x, w)$
	& $C(x, w) = y$
	\rowend
\hline
\end{gadgettabular}
\end{table}
\vspace{-2em}
