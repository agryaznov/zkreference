%%% Luis B: ADD-ON TO INITIATE PART WITH TABLES OF NOTES, IN THE SAME DOCUMENT, 
%%% (CHANGED TO LANDSCAPE) TO CROSS-REFERENCE FEEDBACK NOTES AND CORRESPONDING CHANGES
%%% Appearing vs. not-appearing is controlled with the bit flag boolShowFeedbackNotes in the main file

%%% CHANGE PAGE FORMAT MID-DOCUMENT
\KOMAoption{paper}{landscape}%
\typearea{12}% sets new DIV ??
\recalctypearea
\newgeometry{margin=.65in,footskip=.25in}


\def\mytextfooter{Tables of contribution descriptions}
\fancypagestyle{lscapetablecontribs}{
	\fancyhf{}\lfoot{\mytextfooter}\rfoot{\thepage}
}

%%%%%%%%%%%%%%%%%%%%%%%%%%%%%%%%%%%%%%%%%		
\pagebreak
\thispagestyle{lscapetablecontribs}%
\pagestyle{lscapetablecontribs}%

\def\tempTitle{Tables of contribution descriptions v0.1 \tops{$\rightarrow$}{\textrightarrow} v0.2}
\phantomsection\pdfbookmark[0]{\tempTitle}{pdfbkm:table-of-comments}\label{app:table-comments}
~\hfill\scalebox{2}{\textbf{\tempTitle}}\hfill~


\vspace{1em}
The following pages describe contributions integrated in the process of upgrading the draft reference document from version 0.1 (dated 2019-04-11, available during the 2nd ZKProof Workshop) to version 0.2.


%%%%%%%%%%%%%%%%%%%%%%%%%%%%%%%%%%%%%%%%%%%%%%
\def\tmpSecTitle{Explanation of the tables of contributions}
\section*{\pdfbookmark[2]{\tmpSecTitle}{pdfbkm:explain-table-contributions}\tmpSecTitle}

Each table describes proposed contributions and corresponding edits in comparison with the baseline version 0.1, in order to achieve version 0.2.
Each table, indexed as \letterIndexPropContrib$x$ (where $x$ is an integer), corresponds to a \href{https://github.com/zkpstandard/zkreference/issues}{GitHub issue} 
(GI$y$, where $y$ is an integer) describing proposed contributions --- see \myurl{https://github.com/zkpstandard/zkreference/issues}.
However, compared with GitHub, the description here may have been adjusted for a better explanation and cross-referencing of the actual edits made in the document.
Each table has a header as follows:

\begingroup
\vspace{-1em}
\renewcommand{\thecntContrib}{\letterIndexPropContrib$x$}
\beginlongtabIssue\hline\rowcolor{LLyellow}\headerForIssue\rowend\hline\endlongtable %\end{longtable}
\endgroup

From left to right, the columns represent:

\begin{itemize}

\item \bm{\#}:
A consecutive positive integer, used to count all described items of contribution

\item \textbf{Item id:} 
An index (e.g., \ref{it:editorial:add-abstract}) of the contribution item, with a numbering subordinate to index (e.g., \hyperref{issue:editorial-structural}) the table where it belongs.
	
\item \textbf{Location:} 
A hint about the location (e.g., section number) of the edits, either in the old or in the new document.

\item \textbf{\colNamePropContrib\ \letterIndexPropContrib$x$: \emph{short title}}: 
An identifier \textbf{\letterIndexPropContrib$x$} (with integer $x$) of the contribution \textbf{d}escription, and a title of the issue / contributions.

\item \textbf{Related}: Related references, such as references (GI$x$) to GitHub issues, and/or ids of other contribution items.

\item \textbf{Changes made}: 
Contextual information about the proposed contribution, as well as a high level description of the changes in the document.

\item \textbf{Edit id}:
Index (or possibly several indices) of the edits (E$y$, with integer $y$) made in the document.
Across the document, changes will be marked in the right margin with this index, so that the reader can hyperlink it directly to the description of the contribution, i.e., to an explanation of why the change was made.
\looseness=-1

\end{itemize}

%%% LIST OF CONTRIBUTIONS
\clearpage\phantomsection
\pdfbookmark[1]{\listcontributionname}{pdfbkm:list-of-contributions}

{
\setlength{\cftbeforeloctitleskip}{2em}
\setlength{\cftsecindent}{0em}
\setlength{\cftsecnumwidth}{0em}
}
\renewcommand\cftloctitlefont{\bfseries\LARGE}
\addtocontents{loc}{\vspace*{-3em}}
\addtocontents{loc2}{\vspace*{-3em}}
\listofcontributions
